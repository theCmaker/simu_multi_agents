\chapter{Conclusion}

    La réalisation de ce projet nous a permis d'acquérir plus de pratique du C++, et d'apprendre à penser un programme en équipe. Egalement, la répartition du travail a eu son importance, et l'écriture d'un diagramme de Gantt nous a permis de jalonner toute la période de développement de différents repères afin de mener à bien la réalisation de ce projet, et de minimiser le nombre de bugs à corriger.

    Nous sommes satisfaits de notre travail, même si nous aurions souhaité produire un programme encore plus abouti en termes de paramètres ajustables pour la simulation.

    Nous avons été confrontés à différents problèmes lors du développement, notamment pour garantir l'``étan- chéité'' de la mémoire (Suppression du singleton, suppression des planètes remplacées lors des attaques...). D'autres problèmes se sont présentés lors de la suppression des planètes colonisés n'ayant pas encore joué, nous devions supprimer toutes leurs demandes de fonds pour empêcher des erreurs de segmentation qui arrivaient bien plus tard dans l'exécution et c'est pourquoi elles ont été difficiles à détecter. 
    Nous avons à plusieurs reprises été victimes d’une erreur lors de suppression d’un objet pointé par un itérateur qui parcourait un conteneur. En effet, la méthode \texttt{erase(iterator)} retourne un itérateur sur l’objet suivant celui qui est supprimé dans le conteneur, or nous ne récupérions pas cette valeur, et ainsi cela provoquait une erreur de segmentation au tour de boucle suivant.

    Nous avons été sensibles à la qualité du travail attendu, c’est pourquoi nous avons apporté beaucoup de soin dans le codage et la documentation de notre programme. Dans cette optique nous avons aussi proposé plusieurs exemples de simulation, à 2, 3 et 4 factions afin d’en observer le comportement et ainsi pouvoir ajuster certains paramètres (parfois borner certaines valeurs) de manière à obtenir des résultats plus réalistes.
    
    Enfin, nous avons revu notre diagramme de Gantt initial pour finalement respecter celui présenté ci-après, plus réaliste compte-tenu de la quantité de travail à fournir.

    \begin{center}
        \begin{ganttchart}[vgrid,hgrid]{1}{10}
\gantttitle{Séances}{10} \\
\gantttitlelist{1,...,10}{1} \\
\ganttgroup{Conception de base}{1}{5} \\
\ganttbar{Création de la grille "Monde"}{1}{1} \\
\ganttbar{Création des agents de base}{1}{1} \\
\ganttbar{Tirage des agents à exécuter}{2}{2} \\
\ganttbar{Planète mère et colonies}{3}{5} \\
\ganttbar{Faction}{3}{5} \ganttnewline
\ganttgroup{Simulation}{5}{9} \\
\ganttbar{Création du système économique}{5}{6} \\
\ganttbar{Création des désirs des factions (ordres etc)}{7}{9} \ganttnewline
\ganttgroup{Bonus !}{9}{10} \\
\ganttbar{Spécialisation de faction}{9}{9} \\
\ganttbar{Interface graphique}{10}{10}
\end{ganttchart}
    \end{center}

    \vfill 
    Notre code source est disponible sur la plateforme GitHub à l'adresse :
    \begin{center}
      \href{https://github.com/theCmaker/simu\_multi\_agents}{https://github.com/theCmaker/simu\_multi\_agents}
    \end{center}
    \vfill \null