\chapter{Résultats}
Ces résultats, bien que non utilisables puisque notre simulation a été créée avec des valeurs empiriques permettent toutefois de quantifier les valeurs et les probabilités liées aux agents de la simulation.

Les significations des cases sont les suivantes:

\begin{itemize}
\item Death : Nombre de tours avant la mort de la faction
\item Money : Argent à la mort de la faction
\item Prod : Argent produit par la faction
\item Spent : Argent dépensé par la faction
\item Nb colonies : Nombre de colonies à la mort de la faction
\item Nb attacks : Nombre d’attaques exécutées par la faction
\item Nb success : Nombre d’attaque réussies
\item Nb fails : Nombre d’attaques échouées\\

\item ML x : Position x de la planète mère de la faction
\item ML y : Position x de la planète mère de la faction
\item ML natdef : Défense naturelle de la planète mère (à l’inititialisation)
\item ML natprod : Production naturelle de la planète mère (à l’inititialisation)
\item ML findef : Défense finale de la planète mère
\item ML finprod : Production finale de la planète mère
\end{itemize}

\section{Résultats de cent simulations avec deux factions}
\subsection{Moyenne faction gagnante}

\begin{tabular}{|c|c|c|c|c|c|c|c|}
	\hline Death & Money & Prod & Spent & Nb colonies & Nb attacks & Nb success & Nb fails \\ 
	\hline 1028,34 & 156366,954 & 236889,458 & 180007,563 & 240,73 & 6251,58 & 5596,68 & 654,9 \\
	\hline
\end{tabular}

\begin{tabular}{|c|c|c|c|c|c|}
	\hline ML x & ML y & ML natdef & ML natprod & ML findef & ML finprod \\ 
	\hline 8,96 & 9,47 & 38,16 & 8,27 & 89,56 & 25,39 \\
	\hline
\end{tabular}

\subsection{Moyenne autres factions (éliminées)}

\begin{tabular}{|c|c|c|c|c|c|c|c|}
	\hline Death & Money & Prod & Spent & Nb colonies & Nb attacks & Nb success & Nb fails \\ 
	\hline 1026,34 & 131796,532 & 134702,934 & 156382,954 & 82,48 & 6099,82 & 5487,23 & 612,59 \\
	\hline
\end{tabular}

\begin{tabular}{|c|c|c|c|c|c|}
	\hline ML x & ML y & ML natdef & ML natprod & ML findef & ML finprod \\ 
	\hline 9,29 & 10,01 & 37,98 & 6,54 & 88,55 & 23,79 \\
	\hline
\end{tabular}

\section{Interprétation}

En règle générale, il semble que les valeurs finales des factions gagnantes sont supérieures aux valeurs des factions vaincues.

Il est intéressant de noter que les défenses naturelles (défense à l’initialisation) des factions est un facteur qui a moins d’importance que la production naturelle (à l’initialisation) car si la moyenne du premier est environ égale à celle d’une faction gagnante contre une faction perdante, le deuxième lui est légèrement plus élevé pour la faction gagnante.

Un autre point important est qu’en moyenne la faction gagnante possède trois fois plus de colonies à la fin de la simulation que la faction perdante.